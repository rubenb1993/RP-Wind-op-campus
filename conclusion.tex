\section{Conclusion and Discussion}
\label{conclusion}
From figure \ref{fig:ldl} it can be observed that around a relative slug length of 0.64, the pattern becomes stable. This is caused by the fact that with a relative slug length of 0.64, a maximum of one bubble is in a branch before the reservoir when another bubble arrives at the junction. Because all the lengths of the bubbles are below the critical length (as can be seen in figure \ref{fig:critical}), the cumulative resistance of the bubble and the branch is higher than if there was no bubble in that branch. In other words, the bubble adds more resistance per unit length than a piece of branch does per unit length. Therefore, the bubble will enter the branch with no bubbles. 
\par
Another thing that can be concluded from both figures \ref{fig:critical} and \ref{fig:ldl} is the fact that with higher velocities, the behaviour of the bubbles is more likely to be stable. This can be attributed to the fact that a maximum gas pressure of 1.85 bar was present at the experimental set-up. From equation \ref{Lslug} it is seen that the slug length is proportional to the ratio of liquid flow rate and gas flow rate. From equation \ref{vbubble} it can be seen that the speed of the bubble is linearly proportional to both the $Q_{gas}$ and $Q_{liquid}$. Because there was a maximum pressure used, the increase in speed is only due to the increase in liquid flow rate. This means that at higher flows, the slug length would be higher, creating a stable pattern. 
\par
The secondary goal of this experiment has also been addressed, as seen in figure \ref{fig:secgoal}. It shows clearly that the presented formula does not give the correct values for the actual speeds of the bubble. A reason for this might be dust and other pollution that cause a bigger pressure drop over the whole system than expected. This lowers the flow rates of both the gas and liquid, which in turn lowers the speed of the bubble. Although the channel has been inspected for pollution, it might also manifest itself in the tubes from the syringe and the gas flow. 
\par
A point that hindered the full potential of this experiment was the fact that a maximum pressure available. If there was a higher maximum pressure available, a bigger range of gas flows could have been explored and the relative bubble lengths could have been larger. This would have enabled more research into the critical length and the validity of the speed formula.
\par
For further research, it could be interesting to look at higher gas flow rates. A higher spread of data points might also give insight in a maximum relative $L_{slug}$. It appears that there is no maximum $\frac{L_{slug}}{L_{branch}}$ for which the behaviour is a perfectly alternating pattern with this amount of data. Different geometries might cause different behaviour as seen in formula \ref{LbubbleD}.