\section{Results}
\label{sec:results}
From the theory it is known that there is a critical length, a length of the bubbles for which they all will enter one branch of the junction, as described in section \ref{theory}. In figure \ref{fig:critical} this theoretical critical length is plotted, along with the data points gathered from this experiment. As can be seen in figure \ref{fig:critical}, all the observed bubble lengths remained below the theoretical limit and none of them exhibited this critical behaviour. 
\begin{figure}[ht]
\centering
\includegraphics[width = 0.8 \textwidth]{critical.jpg}
\caption{A diagram of bubble behaviour at a junction. $\frac{L}{D}$ is the length of the bubble divided by the width of the channel (here 200 $\mu m$) and v is the speed of the bubble. The symbols represent different behaviour in the bubbles. $\circ$ represents a perfect alternating pattern, $\diamond$ represents a different stable pattern (2L$\colon$1R or 2R$\colon$1L), $\Box$ represents irregular behaviour due to coalescence and $\triangle$ represents irregular behaviour in the form of a short sequence of bubbles leaving through one or the other branch.
}
\label{fig:critical}
\end{figure}
\par
Another interesting relationship might be the one between $L_{slug}$ and $v_{bubble}$, as proposed by \cite{bypass}. This way of plotting the data can give an insight to the relation between $v_{bubble}$, $L_{slug}$ and the behaviour of the bubbles. Instead of plotting $L_{slug}$, the relative length of $\frac{L_{slug}}{L_{branch}}$ is plotted to give information about the amount of bubbles present in the channel before the analyzed bubble arrives. From this figure, it is shown that above a value of $L_{slug}/L_{branch}$ of 0.39, stable patterns start to occur, and above a value of 0.64, perfectly alternating patterns occur (with the exception of one pattern observed at $L_{slug}/L_{branch}$ = 1.3). 
\begin{figure}[ht]
\centering
\includegraphics[width = 0.8 \textwidth]{ldl.jpg}
\caption{The behaviour of bubbles at a junction. $L_{slug}/L_{brach}$ is the slug length over the length of the arm (here 2000 $\mu m$) and the velocity in mm/s. The same symbols represent the same behaviour as in figure \ref{fig:critical}.}
\label{fig:ldl}
\end{figure}
A secondary goal of this experiment was to verify the relation given by equation \ref{eq4}. In figure \ref{fig:secgoal} there are 2 plots of $v_{bubble}$ as calculated by the equation and the actual measured $v_{bubble}$. 
\begin{figure}
\centering
\begin{subfigure}{0.45\textwidth}
\centering
\includegraphics[width =\textwidth]{1bar}
\caption{1 bar}
\label{fig:sub1}
\end{subfigure}
\begin{subfigure}{0.45\textwidth}
\centering
\includegraphics[width =\textwidth]{2bar}
\caption{1.85 bar}
\label{fig:sub2}
\end{subfigure}
\caption{The speed determined by the liquid flow rate and the gas flow rate. In \ref{fig:sub1} the gas flow rate was induced by a pressure of 1 bar over the whole system, while in \ref{fig:sub2} the pressure drop was 1.85 bar. In both plots, $+$ is the expected velocity and $\ast$ is the measured bubble velocity.}
\label{fig:secgoal}
\end{figure}