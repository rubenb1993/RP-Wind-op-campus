\section{Theory}
\label{theory}
To understand the behaviour of bubbles better, a basic understanding of hydrodynamics is requried. Hydrodynamics bears many similarities to electricity. A fluidum that flows has a pressure drop  $\Delta P$ over a length, this can be seen as the voltage. The fluidum has a flow rate $Q$, current, and is subject to friction $R$ due to the pipes, resistance. With these similarities the traits of a hydrodynamic system become more apparent. The question: `In which branch will the bubble go?'  is reduced to `Which branch has the greater pressure drop?'. Since the pressure drop $\Delta P$ is key to the analysis, the focus changes to which factors influence pressure drops. \\
\\
The pressure drop due to friction with the channel walls between two points is described by formula \ref{Pchannel}:
\begin{equation}
\protect\label{Pchannel}
\protect\Delta P =\protect\frac{k \protect\mu L U}{2D^2}
\end{equation}  
In formula \ref{Pchannel}: $\Delta P$, the pressure drop in $Pa$, is dependant on; $\mu$ the viscosity in $Pa \cdot s$, $L$ the distance between two points in $m$, $D$ the diameter of the channel in $m$, and $U$ the average velocity in the channel in $\frac{m}{s}$. The system geometry influences the constant $k$: for a round channel $k = 64$ and for a square channel $k=56.91$. 
Applying the before found analogy with electricity and assuming that $\Delta P$ is of the form: $\Delta P = R U$ results in formula \ref{Rchannel}:
\begin{equation}
R= \frac{\Delta P}{U} = \frac{k \mu L}{2D^2}
\label{Rchannel}
\end{equation}
Rewriting formula \ref{Pchannel} yields $U$ as described in formula \ref{Speedchannel}.
\begin{equation}
U= \frac{2\Delta P D^2}{k \mu L}
\label{Speedchannel}
\end{equation}
Continuing the application of the analogy with electricity the flow rate can be calculated with the formula \ref{flowrate}
\begin{equation}
Q =U A = \frac{2 \Delta P D^4}{k \mu L}
\label {flowrate}
\end{equation}
With A being the surface of the channel in $m^2$.
For a symmetric system the resistance introduced by the channel walls can be represented by figure \ref{subfig:C}.
\begin{figure}
\centering
\begin{subfigure}{0.45\textwidth}
\centering
\includegraphics[width =\textwidth]{C.png}
\caption{Resistances of the channel walls.}
\label{subfig:C}
\end{subfigure}
\begin{subfigure}{0.45\textwidth}
\centering
\includegraphics[width = \textwidth]{CB.png}
\caption{Resistances of the channel walls and bubble in the chip.}
\label{subfig:CB}
\end{subfigure}
\caption{The resistances of the chips as represented in resistances in an eletric circuit. in \ref{subfig:C} the resistance of only the channel walls is represented, while in \ref{subfig:CB} the resistance of the channel walls and a bubble is represented. \cite{handout}}
\end{figure}
\\
\\
The pressure drop of a bubble through a square channel can be written as formula \ref{Pbubble} 
\begin{equation}
\Delta P = \frac{7.16}{3} \frac{\gamma}{D} \left(\frac {3\mu v}{\gamma}\right)^\frac{2}{3}
\label{Pbubble}
\end{equation} 
with $\gamma$ the interfacial tension between gas and liquid in $N$ and $v$ the velocity of the bubble in $\frac{m}{s}$.\\
\\
The resistance introduced by the bubble can be found in the same way as with \ref{Rchannel} and assuming that $\Delta P$ is of the form: $\Delta P = R v$  resulting in formula \ref{Rbubble}:
\begin{equation}
R = \frac{\Delta P}{v} = \frac{7.16}{3} \frac{\gamma}{D} \left(\frac{3 \mu}{\gamma}\right)^\frac{2}{3} v^{-\frac{1}{3}}
\label{Rbubble}
\end{equation}
Now assume that a bubble has entered the left branch, the resistances in the system now become as represented in figure \ref{subfig:CB}.\\
\\
Examing formula \ref{Rbubble} yields that $R_{bubble}$ is dependant on $v_{bubble}$ and $L_{bubble}$ and can possibly be smaller than $R_{channel}$ for the same length.  By combining formula \ref{Rchannel} and \ref{Rbubble} it becomes apparent that there should exist a $L_{bubble}$ that forces all bubbles into a certain branch due to $R_{branch} < R_{channel}$ of said branch. This relationship is described in formula \ref{Lcritical}.
\begin{equation}
\frac{L_{bubble}}{D} = \frac{7.16}{3} \frac{\gamma}{v^{\frac{1}{3}}}{\left(\frac{3\mu}{\gamma}\right)}^{\frac{2}{3}}\frac{2}{k\mu}
\label{Lcritical}
\end{equation}
It becomes apparent that bubbles introduce a new, perhaps unwanted, property: memory. The more bubbles found in a branch of the chip the higher the resistance and thus the greater the pressure drop. To negate this new property a reservoir is implemented as shown in figure \ref{fig:reservoir}. The reservoir works by skewing the pressure drop in favor of the branch without a drop in it. It also shortens the memory of the system due to a decrease in length of the branches.\\
\\
\begin{figure}[ht]
\centering
\includegraphics[width = 0.5 \textwidth]{reservoir.png}
\caption{An example of a reservoir used to remove the memory of a system \cite{handout}}
\label{fig:reservoir}
\end{figure}
The $v_{bubble}$, $L_{bubble}$ and $L_{slug}$ can be found respectively with formula \ref{vbubble}, \ref{LbubbleD} and \ref{Lslug} as described in \cite{handout}:
\begin{equation}
v_{bubble}\approx \frac{Q_{liquid}+Q_{gas}}{D^2}
\label{vbubble}
\end{equation}

\begin{equation}
\frac{L_{bubble}}{D} \approx 1 + \alpha\frac{Q_{gas}}{Q_{liquid}}
\label{LbubbleD}
\end{equation}

\begin{equation}
\label{Lslug}
L_{slug} \approx L_{bubble}\frac{Q_{liquid}}{Q_{gas}}
\end{equation}
$\alpha$ is an unknown dimensionless coefficient dependant on the geometry of the junction.