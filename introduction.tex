\section{Introduction}
A way of studying reactive agents is through the use of `laboratories on chips', microfluidic systems consisting of fluid channels and devices in which the flow of fluids can be controlled very precisely. For experiments on reactive gases, it is useful to create small bubbles of reactive gas that travel in these channels while a propulsion liquid is present. Through the control of the fluids, the volume and reaction time can be managed precisely. Another advantage is that the total volume of the reagents is at least a million times smaller than the reaction tubes commonly found in laboratories. 
\par
For analytical reasons, it can be interesting to split the streams of bubbles into one or more microchannels, in which different actions can be performed on the gas. The splitting of these bubbles has been shown to not be stochastic but absolutely determined by the channel configuration and flow rates, as shown in \cite{multiphase}.
\par
The main goal of this research is to study the influence of the length of the bubbles($L_{bubble}$), the distance between the bubbles($L_{slug}$), and the velocity of the bubbles($v_{bubble}$) on the distribution of bubbles over two branches at a Y junction. As mentioned in \cite{bifurcation}, there are three possible ways for the bubbles to distribute themselves; the bubble will either go into the left or right branch, or it will break. The theory behind this ``choice'' is is treated in section \ref{theory}. 
\par
A secondary goal is to verify the given formula \ref{eq4}
\begin{equation}
\label{eq4}
v_{bubble} \approx \frac{Q_{liquid} + Q_{gas}}{D^{2}}
\end{equation}\\
Where $Q_{liquid}$ is the liquid flow rate, $Q_{gas}$ the gas flow rate and $D$ the width and height of the microchannel. This formula is stated in the \cite{handout}. How this is researched will be described in section \ref{expsetup}, while the conclusion of this verification can be found in section \ref{conclusion}. \\