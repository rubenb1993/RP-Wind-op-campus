\section{Introduction}
\label{intro}
The Flatiron building, as seen in figure~\ref{fig:flatiron}, is the iconic Manhattan skyscraper shaped like a right triangle. Clasped between 5th Avenue and Broadway, with Madison Square Park just north-east of him, there is a lot of open space around this building. If the wind comes from the north, it will be forced through an ``alley'', creating a windtunnel effect around the Flatiron building. Legend has it that men would hang out out at the corner to watch the wind blowing women's dresses up so that they could see their ankles~\cite{dresses}. This is also shown on a postcard from the early 20th century (figure \ref{fig:postalcard}), showing a man being blown away by the wind and a woman's skirt being blown up by the wind.  \\
\indent The main goal of this research is to see if it was actually the geometry of the building creating the updraft. This will be done by simulating the building and surrounding buildings in an in-house built CFD program made for CFD analysis for urban areas. \\
\indent This report will first discuss the theory and numerical models of the CFD simulation, after which it describes the cases and their results, and furthermore these results will be discussed and a conclusion about the billowing of the skirts will be made.
\begin{figure}[h!]
\centering
\includegraphics[width = 0.3 \textwidth]{flatiron.jpg}
\caption{A picture of the famous Flatiron building}
\label{fig:flatiron}
\end{figure}
\begin{figure}[h!]
\centering
\includegraphics[width = 0.3 \textwidth]{postcard.jpg}
\caption{A postcard by an unknown artist displaying the unpredictable winds and billowing skirts around the Flatiron building \cite{postalcard}.}
\label{fig:postalcard}
\end{figure}