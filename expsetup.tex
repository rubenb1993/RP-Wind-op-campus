\section{Experimental Setup}
\label{expsetup}
The primary goal of the experiment is to determine the behaviour of bubbles based on the $v_{bubble}$, $L_{bubble}$ and $L_{slug}$. Secondary goals include if the described theoretical relation between the volumetric flows of gas and liquid is true and if there is a critical $\frac{L_{bubble}}{D}$. \\
\\
The setup consists of a specifically fabricated microfluidic chip with a channel height of  $142 ~\mu m$ and channelwidth of $200~ \mu m$, a pressure regulated gassupply, a pump operated syringe which functions as liquidsupply and HFE 7500 is used as liquid. An ip-camera is used to record the behaviour, the framerate of the camera is set at 120 fps. In the setup the gas flow rate $Q_{gas}$ and the liquid flow rate $Q_{liquid}$ can be regulated to achieve desired $L_{bubble}$ and $v_{bubble}$ . In the setup only the gaspressure can be controlled with the help of a pressure regulating valve. The corresponding $Q_{gas}$ can be found by applying formula \ref{flowrate}. To use formula \ref{flowrate} the combined resistance of the chip and gassupply is needed. By utilising the analogy with electricity calculating the total resistance of the chip and tubing simplifies to two resistances in series: $R_{combined} =R_{gas}+R_{chip}$. Calculating the resistances with formula \ref{Rchannel} shows that $R_{gas}\gg R_{chip}$. This means that $R_{combined}$ becomes  $R_{combined}=R_{gas}$. The Liquidflow is managed by setting the syringepump to the desired levels.\\
\\
The chip has the layout as described in figure \ref{fig:layout}.\
\begin{figure}[ht]
\centering
\includegraphics[width = 0.8 \textwidth]{layout.png}
\caption{The layout of the chip used in the experiments.\cite{handout}}
\label{fig:layout}
\end{figure}
\\
\\
The $L_{bubble}$ and $v_{bubble}$ are determined experimentally by analyzing video of the behaviour of the system. To aid in the analysis of the video the program ImageJ was used. The program allows for frame by frame analysis of the video and has built in tools for determing length of lines expressed in pixels see image \ref{subfig:Length}. For calibration purposes the width of the channel was used as a baseline see figure \ref{subfig:Width}. Based on the amount of frames required for the bubble to travel a known length and the framerate, the $v_{bubble}$ can be calculated see figure \ref{fig:S}. With the help of these tools it becomes possible to look at the behaviour of the bubbles; if the theoretical ratio of volumetric flows of gas and liquid as described in formula \ref{vbubble} and if there is indeed a critical $\frac{L_{bubble}}{D}$.

\begin{figure}
\centering
\begin{subfigure}{0.45\textwidth}
\centering
\includegraphics[width =\textwidth]{Length.png}
\caption{Length measurement}
\label{subfig:Length}
\end{subfigure}
\begin{subfigure}{0.45\textwidth}
\centering
\includegraphics[width =\textwidth]{Width.png}
\caption{Calibration}
\label{subfig:Width}
\end{subfigure}
\caption{Figure \ref{subfig:Length} shows how the $L_{bubble}$ was determined with the help of ImageJ, figure \ref{subfig:Width} shows how the camera was calibrated by measuring the width of the channel.}
\label{fig:Lengtht}
\end{figure}

\begin{figure}
\centering
\begin{subfigure}{0.45\textwidth}
\centering
\includegraphics[width =\textwidth]{S1.png}
\caption{Initial frame}
\label{subfig:s1}
\end{subfigure}
\begin{subfigure}{0.45\textwidth}
\centering
\includegraphics[width =\textwidth]{S2.png}
\caption{Last frame}
\label{subfig:s2}
\end{subfigure}
\caption{Figure \ref{subfig:s1} shows the first frame used for determining $v_{bubble}$, while figure \ref{subfig:s2} shows the last frame used. By taking the length traversed by the bubble and the amount of frames $v_{bubble}$ is determined.}
\label{fig:S}
\end{figure}